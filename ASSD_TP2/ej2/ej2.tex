\documentclass[../ASSD_TP2.tex]{subfiles}
\begin{document}
\section*{Síntesis de archivo MIDI}
MIDI(Musical Instrument Digital Interface), tal como su nombre lo indica es un protocolo de comunicacion, orientado a los instrumentos musicales y sintetizadores. Un Archivo MIDI contiene información sobre el instrumento que se esta usando, cuando se toca determinada nota y con que intensidad se lo utiliza. 
\subsection*{Esquema básico de archivo MIDI}
La estructura de estos archivos esta compuesto de la siguiente manera:
\begin{enumerate}
\item Cansion
	\begin{enumerate}
		\item tracks(cada track representa un instrumento)
			\begin{enumerate}
				\item message (Cada mensaje es una acción o configuración que realiza el instrumento)
					\begin{enumerate}
						\item metadata (informacion relevante de cada mensaje)
					\end{enumerate}						
			\end{enumerate}
	\end{enumerate}
\end{enumerate}
Se utilizo la biblioteca mido para python, que realiza el parseo del archivo MIDI, devolviendo un diccionario de tracks.
\subsection*{Mensages relevantes del archivo MIDI}

Hay dos mensajes que se utilizaron para sintetizar el archivo MIDI, NOTE\_ON y NOTE\_OFF. Tal como su nombre lo indica, los mensajes marcan el comienzo y fin de cada nota. Ambos mensajes comparten la misma metadata:
\begin{itemize}
\item tiempo(cuando se realiza la acción)
\item note 
\item velocity (intensidad de la nota)

\end{itemize}
El equivalente al mensaje de NOTE\_OFF es el NOTE\_ON con velocity 0.

\subsection*{Programa implementado}
El problema a resolver es la sintetizacion de un solo track, debido a que un MIDI esta compuesto por tracks, se sintetizan de a uno y posteriormente se suman.
\par El primer paso es armar la linea de tiempo correspondiente a al track, es decir cuendo se ejecuto cada nota y cuanto duro. Para ello se utilizo la función $trackParser (track,tickPerBeat,tempo)$, recive como parametros el track, los tickPerBeat y el temo, devolviendo un arreglo de notas. Cada nota tiene su frecuencia, su tiempo de comiezo y de fien en segundos.
Despues resta asignar a cada nota el instrumento, para ello se utiliza la función $trackGenerator(parsedTrack,instrumento,Fs)$, recive como parametro el vector de notas, el instrumento deseado y la frecuencia de sampleo, devolviendo el track sintetizado con el instrumento deseado.
\par En cuanto a los instrumentos, se utilizaron los desarrollados en los otros  puntos del informe.






\end{document}
